\documentclass[fleqn]{article}

%\pgfplotsset{compat=1.17}

\usepackage{mathexam}
\usepackage{amsmath}
\usepackage{amsfonts}
\usepackage{graphicx}
\usepackage{systeme}
\usepackage{microtype}
\usepackage{multirow}
\usepackage{pgfplots}
\usepackage{listings}
\usepackage{tikz}
\usepackage{dsfont} %Numeros reales, naturales...
\usepackage{cancel}

%\graphicspath{{images/}}
\newcommand*{\QED}{\hfill\ensuremath{\square}}

%Estructura de ecuaciones
\setlength{\textwidth}{15cm} \setlength{\oddsidemargin}{5mm}
\setlength{\textheight}{23cm} \setlength{\topmargin}{-1cm}



\author{David García Curbelo}
\title{Topología}

\pagestyle{empty}


\def\R{\mathds{R}}
\def\Z{\mathds{Z}}
\def\N{\mathds{N}}

\def\sup{$^2$}

\def\next{\quad \Rightarrow \quad}

\begin{document}

    \setcounter{page}{1}
    \pagestyle{plain}

    \begin{center}
        {\large\bf{Topología II}} \\
        \bf{David García Curbelo}\\
        
    \end{center}

    \textbf{Ejercicio 37. } \textit{¿Es todo retracto de un espacio $X$ un retracto de deformación de $X$?}\\

    Claramente el enunciado no es cierto. Para la construcción del contraejemplo basta con tomar un espacio topológico $X$
    con grupo fundamental no trivial y el conjunto $A$ como cualquier punto suyo. Tomemos un ejemplo particular:

    Consideremos $X = \mathbb{S}^1$ y $A = \{x_0\}$, con $x_0 \in X$ cualquier punto de la circunferencia. Consideremos también
    la aplicación
    $$
    \begin{aligned}
        r : & X \longrightarrow A \\
            & x \longmapsto x_0
    \end{aligned}
    $$
    la cual vemos fácilmente que es continua, ya que se trata de la aplicación constante que asigna a todo punto de $X$ el 
    único punto del conjunto $A$. Además, considerando la aplicación inclusión $i_A : A \hookrightarrow X$ tenemos
    $$(r \circ i_A)(x_0) = r(i_A(x_0)) = r(x_0) = x_0 \next r \circ i_A = 1_A$$
    Con lo que concluimos que nuestra aplicación $r$ es una retracción.\\

    %por la misma razón se tiene que $r|_A = \text{Id}_A$, luego tenemos que nuestra aplicación $A$ es un retracto de $X$. \\
    %Tomemos también para nuestro estudio la aplicación inclusión $$i_A : A \hookrightarrow X$$
    

    Supongamos ahora que $A$ es un retracto de deformación de $X$. Si esto fuera así, tendríamos para todo elemento $a \in A$ el isomorfismo
    dado por 
    $$(i_A)_* : \Pi_1 (A, a) \longrightarrow \Pi_1 (X, a)$$
    Y así tendríamos que $\Pi_1 (X, a)$ y $\Pi_1 (A, a)$ son isomorfos. Luego, de la misma manera, como $A = \{x_0\}$, tenemos que
    $$\Pi_1 (X, x_0) \overset{iso}{\cong} \Pi_1 (A, x_0)$$
    Pero esto no es verdad, ya que 
    $$\Pi_1 (X, x_0) \overset{iso}{\cong} \Pi_1 (\mathbb{S}^1, x_0) \overset{iso}{\cong} \Z \quad \quad \quad \quad 
    \Pi_1 (A, x_0) \overset{iso}{\cong} \Pi_1 (\{x_0\}, x_0) \overset{iso}{\cong} \{1\}$$
    Luego hemos llegado a una contradicción, que está motivada por la única suposición que hemos hecho: $A\subset X$ es un retracto de deformación.
    Queda así probado que no todo retracto de $X$ tiene por qué ser un retracto de deformación de $X$.

    %Todo punto es un retracto del espacio que lo contiene. La aplicación constante $r: X \rightarrow \{x_0\}$

\end{document}