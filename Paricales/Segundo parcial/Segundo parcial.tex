\documentclass[fleqn]{article}

%\pgfplotsset{compat=1.17}

\usepackage{mathexam}
\usepackage{amsmath}
\usepackage{amsfonts}
\usepackage{graphicx}
\usepackage{systeme}
\usepackage{microtype}
\usepackage{multirow}
\usepackage{pgfplots}
\usepackage{listings}
\usepackage{tikz}
\usepackage{dsfont} %Numeros reales, naturales...
\usepackage{cancel}
\usepackage{setspace} %Para modificar el interlineado

%\graphicspath{{images/}}
\newcommand*{\QED}{\hfill\ensuremath{\square}}

%Estructura de ecuaciones
%\setlength{\textwidth}{15cm} \setlength{\oddsidemargin}{5mm}
%\setlength{\textheight}{23cm} \setlength{\topmargin}{-1cm}



\author{David García Curbelo}
\title{Topología}

\pagestyle{empty}

%\renewcommand{\baselinestretch}{2} para modificar el interlineado en todo el doc

\def\R{\mathds{R}}
\def\Z{\mathds{Z}}
\def\N{\mathds{N}}
\def\S{\mathds{S}}

\def\sup{$^2$}

\def\next{\quad \Rightarrow \quad}

\begin{document}
    \doublespace

    \setcounter{page}{1}
    \pagestyle{plain}

    \begin{center}
        {\large\bf{Topología II}} \\
        \bf{Segundo parcial 2020}\\
        
    \end{center}

    \textbf{Ejercicio 1. } Sea $X$ un espacio topológico conexo y localmente arcoconexo y supongamos que $(\tilde{X}, \pi)$ es un recubridor universal
    de $X$. Sean $x \in X$ un punto y $\alpha \in \Omega_x(X)$ un lazo de $X$ basado en el punto $x$ y supongamos que $\alpha$ no es homotópico al lazo constante
    $\varepsilon_x$.

    Decidir razonadamente si la siguiente afirmación es cierta o falsa:\\
    \textit{Si $\tilde{\alpha}$ es un levantamiento de $\alpha$ a $(X, \pi)$, entonces $\tilde{\alpha} (0) \neq \tilde{\alpha} (1)$; es decir, 
    $\tilde{\alpha}$ no es un lazo.} 

    \newpage

    \textbf{Ejercicio 2. } Sea $X$ un subespacio topológico de $\R^2$ dado por $X = \S^1 \cup L_1 \cup L_2$, donde $\S^1 = \{(x,y) \in \R^2 : x^2 + y^2 = 1\}$ es la 
    circunferencia unidad, $L_1 = \{(x,0) \in \R^2 : x \in [-2,0] \}$ es el segmento cerrado de extremos $(-2,0)$ y $(0,0)$ y 
    $L_2 = \{(x,0) \in \R^2 : x \in [1,2] \}$ es el segmento cerrado de extremos $(1,0)$ y $(2,0)$.
    \begin{enumerate}
        \item ¿Admite $X$ un recubridor de 31 hojas?
        \item Determinar explicitamente un espacio recubridor de $X$.
        \item Decidir razonadamente si la siguiente afirmación es cierta o falsa: \textit{Si $(\tilde{X}, \pi)$ es un espacio recubridor de $X$ y no es 
                finito, entonces es recubridor universal.}
    \end{enumerate}

    \newpage


\end{document}