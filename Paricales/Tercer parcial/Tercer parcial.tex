\documentclass[fleqn]{article}

%\pgfplotsset{compat=1.17}

\usepackage{mathexam}
\usepackage{amsmath}
\usepackage{amsfonts}
\usepackage{graphicx}
\usepackage{systeme}
\usepackage{microtype}
\usepackage{multirow}
\usepackage{pgfplots}
\usepackage{listings}
\usepackage{tikz}
\usepackage{dsfont} %Numeros reales, naturales...
\usepackage{cancel}
\usepackage{setspace} %Para modificar el interlineado

%\graphicspath{{images/}}
\newcommand*{\QED}{\hfill\ensuremath{\square}}

%Estructura de ecuaciones
%\setlength{\textwidth}{15cm} \setlength{\oddsidemargin}{5mm}
%\setlength{\textheight}{23cm} \setlength{\topmargin}{-1cm}



\author{David García Curbelo}
\title{Topología}

\pagestyle{empty}

%\renewcommand{\baselinestretch}{2} para modificar el interlineado en todo el doc

\def\R{\mathds{R}}
\def\Z{\mathds{Z}}
\def\N{\mathds{N}}
\def\S{\mathds{S}}

\def\sup{$^2$}

\def\next{\quad \Rightarrow \quad}

\begin{document}
    \doublespace

    \setcounter{page}{1}
    \pagestyle{plain}

    \begin{center}
        {\large\bf{Topología II}} \\
        \bf{Tercer parcial 2020}\\
        
    \end{center}

    Consideremos una superficie compacta y conexa con presentación poligonal
    $$P = \langle \{a,b,c,d,e,f,g,h,i,j,k\} ; ackb^{-1}af^{-1}, cdhg^{-1}je^{-1}j^{-1}, fik^{-1}eg^{-1}h^{-1}d^{-1}ib^{-1} \rangle$$
    ¿A qué superficie modelo es homeomorfa? Señale sólo una:

    \begin{enumerate}
        \item La esfera
        \item Suma conexa de 3 toros
        \item Suma conexa de 4 toros
        \item Suma conexa de 5 toros
        \item Suma conexa de 6 planos proyectivos
        \item Suma conexa de 7 planos proyectivos
        \item Suma conexa de 8 planos proyectivos
        \item Ninguna de las anteriores
    \end{enumerate}

    \newpage

\end{document}