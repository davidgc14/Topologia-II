\documentclass[fleqn]{article}

%\pgfplotsset{compat=1.17}

\usepackage{mathexam}
\usepackage{amsmath}
\usepackage{amsfonts}
\usepackage{graphicx}
\usepackage{systeme}
\usepackage{microtype}
\usepackage{multirow}
\usepackage{pgfplots}
\usepackage{listings}
\usepackage{tikz}
\usepackage{dsfont} %Numeros reales, naturales...
\usepackage{cancel}
\usepackage{setspace} %Para modificar el interlineado

\usepackage{pgfplots}
\pgfplotsset{compat=1.7}

%\graphicspath{{images/}}
\newcommand*{\QED}{\hfill\ensuremath{\square}}


%Estructura de ecuaciones
%\setlength{\textwidth}{15cm} \setlength{\oddsidemargin}{5mm}
%\setlength{\textheight}{23cm} \setlength{\topmargin}{-1cm}



\author{David García Curbelo}
\title{Topología}

\pagestyle{empty}

%\renewcommand{\baselinestretch}{2} para modificar el interlineado en todo el doc

\def\R{\mathds{R}}
\def\Z{\mathds{Z}}
\def\N{\mathds{N}}
\def\S{\mathds{S}}

\def\sup{$^2$}

\def\next{\quad \Rightarrow \quad}

\begin{document}
    \doublespace

    \setcounter{page}{1}
    \pagestyle{plain}

    \begin{center}
        {\large\bf{Topología II}} \\
        \bf{Primer parcial 2020}\\
        
    \end{center}

    Sea $X_0$ el subespacio topológico de $\R^3$ dado por la unión $\S^2_{(0,0,0)} \cup \S^2_{(3,0,0)} \cup L$, donde $\S^2_{(0,0,0)}$ es la esfera 
    de centro $(0,0,0)$ y radio 1, $\S^2_{(3,0,0)}$ es la esfera de centro $(3,0,0)$ y radio 1, y 
    $L = \{(x,0,0) \in \R^3 \thinspace : \thickspace x \in [2,4]\}$ es el segmento que conecta los polos oeste y este de la esfera $\S^2_{(3,0,0)}$.
    Observa que $X_0$ no es conexo, ni arcoconexo. Consideremos en $X_0$ la relación de equivalencia en la que dos puntos $p,q \in X_0$ están relacionados
    si y sólo si pasa una de las tres siguientes posibilidades:
    \begin{enumerate}
        \item $p = q$
        \item $\{p,q\} = \{(0,0,1), (3,0,1)\}$; es decir, identificamos polo norte de una esfera con polo norte de la otra.
        \item $\{p,q\} = \{(0,0,-1), (3,0,-1)\}$; es decir, identificamos polo sur de una esfera con polo sur de la otra.
    \end{enumerate}

    Definimos $X$ como el espacio topológico cociente de $X_0$ por esta relación de equivalencia. Justificar que $X$ es arcoconexo y calcular su grupo 
    fundamental (en el punto que se quiera).

    \noindent
    CONSEJO: Haz un dibujo de un subespacio topológico de $\R^3$ que sea homeomorfo a $X$ y calcula el grupo fundamental de ese subespacio (basta con un 
    dibujo, no es necesario dar una descripción explícita del subespacio). 

    \newpage


\end{document}